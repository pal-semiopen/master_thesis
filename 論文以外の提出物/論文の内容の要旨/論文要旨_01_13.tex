\documentclass[dvipdfmx]{article}
\usepackage[utf8]{inputenc}

\usepackage{color}
\usepackage[T1]{fontenc}
\usepackage{geometry}
\usepackage{amsmath,amssymb,amscd,amsthm}
\usepackage{amsfonts}
\usepackage{comment}
\usepackage{mathrsfs} %\mathscrの使用
\geometry{a4paper}

%\usepackage[frenchb]{babel}

\usepackage{graphicx}  % \rotateboxの使用
%\usepackage[all]{xy} % \xymatrixの使用、図式用
\usepackage{tikz-cd} % tikzcd環境の使用、図式用
\usepackage{quiver} %quiver.appの使用
\usepackage{adjustbox} %\adjustboxの使用、図式サイズの調整用

\usepackage[dvipdfmx]{hyperref} %ハイパーリンクの使用
\usepackage{pxjahyper} %hyperrefの補助
\hypersetup{%
 setpagesize=false,%
 %bookmarks=true,%
 %bookmarksdepth=tocdepth,%
 bookmarksnumbered=true,%
 colorlinks=true, %リンクを色文字にするかどうか。しないときは色枠線で囲う
 linkcolor={[rgb]{0.1,0.1,0.3}}, %目次などの通常のリンクの色。
 urlcolor={[rgb]{0,0.8,0.8}}, %参考文献などの外部リンクの色。
 pdftitle={},%
 pdfsubject={},%
 pdfauthor={},%
 pdfkeywords={}}

\newcommand{\hang}[1]{\noindent \settowidth{\hangindent}{#1}#1}

\newcommand{\card}[1]{ |#1| }
%\newcommand{\N}{\mathbb{N}}
\newcommand{\Z}{\mathbb{Z}}
\newcommand{\Q}{\mathbb{Q}}
%\newcommand{\R}{\mathbb{R}}
\newcommand{\powerset}[1]{\mathfrak{P}(#1)}
\renewcommand{\P}{\mathcal{P}}
\renewcommand{\Im}{\mathrm{Im}}
\newcommand{\ideal}{\mathrm{Ideal}}
\newcommand{\open}{\mathrm{open}}
\newcommand{\closed}{\mathrm{closed}}
\newcommand{\Hom}{\mathrm{Hom}}
\newcommand{\End}{\mathrm{End}}
\newcommand{\Fct}{\mathrm{Fct}}
\newcommand{\colim}{\mathrm{colim}}
\newcommand{\for}{\mathrm{for} \ }
\newcommand{\maru}[1]{\textcircled{\scriptsize #1}}
\newcommand{\norm}[1]{\|#1\|}
\newcommand{\abs}[1]{ |#1| }
\newcommand{\inp}[2]{\langle#1,#2\rangle}
\newcommand{\conj}[1]{\overline{#1}\ }
\newcommand{\set}[1]{\left\{ #1 \right\}}
\newcommand{\gen}[1]{\langle#1\rangle}
\renewcommand{\phi}{\varphi}
\renewcommand{\epsilon}{\varepsilon}
\renewcommand{\O}{\mathcal{O}}
\newcommand{\U}{\mathcal{U}}
\newcommand{\V}{\mathcal{V}}
\newcommand{\F}{\mathcal{F}}
\newcommand{\B}{\mathcal{B}}
\newcommand{\A}{\mathcal{A}}
\newcommand{\C}{\mathcal{C}}
\newcommand{\D}{\mathcal{D}}
\newcommand{\I}{\mathcal{I}}
\newcommand{\E}{\mathcal{E}}
\newcommand{\M}{\mathcal{M}}
\newcommand{\N}{\mathcal{N}}
\newcommand{\R}{\mathcal{R}}
\newcommand{\Lam}{\Lambda}
\newcommand{\LAM}{\Lam}
\newcommand{\lam}{\lambda}
\renewcommand{\l}{\lambda}
\newcommand{\m}{\mu}
\newcommand{\n}{\nu}
\newcommand{\y}{\eta}
\newcommand{\e}{\epsilon}
\renewcommand{\d}{\delta}
\newcommand{\closure}[1]{\overline{#1}}
\newcommand{\0}{\emptyset}
\renewcommand{\o}{\circ} %射の合成(逆)
\newcommand{\comp}{\triangleright} %射の合成(順)
\newcommand{\app}{\blacktriangleright\ } %写像への代入(順)
\newcommand{\subgroup}{\leq}
\newcommand{\acts}{\curvearrowright}
\newcommand{\tensor}{\otimes}
\renewcommand{\bar}[1]{\overline{#1}}
\newcommand{\Cat}{\mathbf{Cat}}
\newcommand{\Set}{\mathrm{Set}}
\newcommand{\Mon}{\mathrm{Mon}}
\newcommand{\Gp}{\mathrm{Gp}}
\newcommand{\Ab}{\mathrm{Ab}}
%\renewcommand{\Vec}{\mathrm{Vec}}
\newcommand{\Mod}{\mathrm{Mod}}
\newcommand{\Ring}{\mathrm{Ring}}
\newcommand{\Alg}{\mathrm{Alg}}
%\newcommand{\}{\mathrm{}}
%\renewcommand{\/}{\backslash}
\renewcommand{\-}{\mathchar`-}
\newcommand{\BOOL}{\mathrm{BOOL}}
\newcommand{\Top}{\mathrm{Top}}
\newcommand{\STONE}{\mathrm{STONE}}
\newcommand{\Forget}{\mathrm{Forget}}
\newcommand{\Free}{\mathrm{Free}}

\newcommand{\Eta}{\mathit{H}}
\newcommand{\Epsilon}{\mathit{E}}
\newcommand{\rmin}{{\ \mathrm{in} \ }}




\newcounter{mynum}
\renewcommand{\themynum}{\arabic{mynum}.}
\newcommand{\myitem}{\refstepcounter{mynum}\mbox{\themynum}}





\newcommand{\map}[5]{
\[
\begin{tikzcd}[ampersand replacement=\&]
#1 \&[-1.05cm]: \&[-1.03cm] #2 \arrow[r] \arrow[d,phantom, "\ni", sloped]  \& #3\arrow[d,phantom, "\ni", sloped] \\[-0.4cm]
      \&            \&              #4 \arrow[r,mapsto]                                            \& #5
\end{tikzcd}
\]
}

\newcommand{\doublemap}[9]{
\[
\begin{tikzcd}[row sep=0.4cm, ampersand replacement=\&]
#1\&[-1cm] : \&[-1cm] #2 \arrow[r] \arrow[d,phantom, "\ni", sloped] \& #3  \arrow[d,phantom, "\ni", sloped] \\
      \& \&#4 \arrow[r,mapsto] \& #5\&[-1.06cm] :\&[-1.04cm] #6 \arrow[d,phantom, "\ni", sloped]\arrow[r] \& #7 \arrow[d,phantom, "\ni", sloped]\\
       \& \&                        \&       \&       \&         #8\arrow[r,mapsto] \&  #9
\end{tikzcd}
\]
}



\newcommand{\fct}[9]{	%covariant functor

\[
\begin{tikzcd}[column sep=1.2cm, ampersand replacement=\&]
{#1}	\&[-1.05cm]:	\&[-1.03cm] {#2} \arrow[r] \arrow[d,phantom, "\ni", sloped]	\& {#3}\arrow[d,phantom, "\ni", sloped] \\[-0.4cm]
	\&			\&			{#4}\arrow[r,mapsto]\arrow[d,"{#5}"{name=f}]	\& {#7}\arrow[d,"{#8}"'{name=g},shift left=0.5ex]\\
	\&			\&			{#6}\arrow[r,mapsto]						\& {#9}\arrow[mapsto,from=f,to=g]
\end{tikzcd}
\]
}











% varLambdaの活用法考えよう








%\title{論文の内容の要旨}
%\author{箕浦 晴弥}
%\date{2019/10/10}

%\setcounter{tocdepth}{4}


\renewcommand{\thesection}{\S \ \arabic{section}}






%%ここからこの論文中でのみ使うコマンド
\newcommand{\SMC}{対称モノイド圏}
\newcommand{\bSMC}{\mathbf{SMC}}
\newcommand{\DMC}{\RC} %Diagonaled Monoidal Category}
\newcommand{\RC}{{関連圏}}
\newcommand{\subRC}{{関連部分圏}}
\newcommand{\DMCs}{\RCs} %Diagonaled Monoidal Categories}
\newcommand{\RCs}{{関連圏}}
\newcommand{\bDMC}{\mathbf{DMC}}
\newcommand{\Rf}{{関連函手}}
\newcommand{\AC}{{アフィン圏}}
\newcommand{\CC}{デカルト圏}
\newcommand{\CCs}{デカルト圏}
\newcommand{\bCC}{\mathbf{CC}}
\newcommand{\calLambda}{\mathcal{B}}
\newcommand{\mono}{{\mathrm{mono}}}
%\newcommand{   }{}
%\newcommand{   }{}
\newcommand{\topic}[1]{\textcolor{black}{#1}}
\renewcommand{\simeq}{\cong}
\newcommand{\Ob}[1]{\mathrm{Ob}(#1)}
\newcommand{\compose}[2]{#1 \comp #2}
\renewcommand{\baselinestretch}{1.5}
%%ここまでこの論文中でのみ使うコマンド





\begin{document}

\centerline{\huge{論文の内容の要旨}}
\ \\

{\Large\underline{修士論文題目}\\

\underline{関連圏からのデカルト圏の普遍的再構成}}

\ \\

\centerline{\Large 氏名 箕浦 晴弥}

\ \\

\section{オリジナルのイントロ}
\topic{対称モノイド圏論は、ホモトピー論を出自としてMacLaneにより創始された(\cite{maclane1963natural})。}
モノイド圏論では、圏に函手的なモノイド演算が乗った構造であるモノイド圏を扱う。対称モノイド圏論は、その中でも特にモノイド演算が(同型を除いて)可換な対称モノイド圏を扱う分野である。
これは直積でない「積」と呼ばれるもの、例えばテンソル積やスマッシュ積などの一般化である。

\topic{対称モノイド圏論は、圏論の中では比較的弱い構造を扱う理論である。}
他の圏論の多くの分野、例えばトポス理論やアーベル圏論では、全ての有限直積があること、即ちデカルト圏であることは大前提として、その上にさらに構造を加えていく。対照的に、対称モノイド圏論においてはデカルト圏が最も強い構造である:対称モノイド圏に構造を付加してデカルト圏より強い構造になるならば、それは単にデカルト圏の理論で扱えばよい。即ち、対称モノイド圏論とは「対称モノイド圏とデカルト圏の中間構造たちを調べる分野」であると言うことができる。




\topic{対称モノイド圏論の視点で見れば、デカルト圏とは次のような構造だとみなせる:}デカルト圏とは、対称モノイド圏 $\C$と自然同型$\phi_{X,Y}:\C(-,X)\times\C(-,Y)\simeq\C(-,X\tensor Y)$の組$(\C,\phi)$のことである。
\topic{この同型を弱めて、対称モノイド圏$\C$によい自然変換$\C(-,X)\times\C(-,Y)\to\C(-,X\tensor Y)$を付加した構造や、対称モノイド圏$\C$によい自然変換$\C(-,X)\times\C(-,Y)\gets\C(-,X\tensor Y)$を付加した構造を考える。}これらはそれぞれ\topic{関連圏(relevance category)} / \topic{アフィン圏(affine category)}と呼ばれるものと一致し、対称モノイド圏とデカルト圏の中間構造として知られている(\cite{Pet02})。
関連圏 / アフィン圏 は、対称モノイド圏にそれぞれ「対角射」「終対象への標準射」を付加したものとして定義される。即ち、関連圏とは対称モノイド圏$\C$とよいモノイド自然変換$\Delta : Id_{\C}\to -\tensor -$の組であり、アフィン圏は対称モノイド圏$\C$であって単位対象が終対象であるようなものである(ただし、ここで函手$-\tensor-$は各対象$X$に$X\tensor X$を対応させる自己函手のことである)。
関連圏には例えば集合と単射の圏 $\Set^{\mono}$ などがある。アフィン圏には例えば凸空間の圏 $\mathrm{Conv}$ などがある。



\topic{\RC や\AC は、対称モノイド圏と\CC のよい中間構造と思える。}具体的には、対称モノイド圏に「対角射」を追加すれば\RC に、「終対象」を追加すれば\AC になるし、\CC は\RC に「終対象」を追加したものと見なせ、また\AC に「対角射」を追加したものとも見なせる。\topic{ゆえに、これら中間構造を調べることは対称モノイド圏と\CC の関係性を調べるのに有用である。}例えば、与えられた対称モノイド圏 $\M$から普遍的な\CC $CC[\M]$を構成する問題(直積を付加する問題)は、与えられた対称モノイド圏 $\M$から普遍的な\RC $RC[\M]$を構成する問題(対角射を付加する問題)と与えられた\RC $\R$から普遍的な\CC $CC[\R]$を構成する問題(終対象を付加する問題)とに分割できる。
実際\AC については、アフィン論理への関心も相まって多くの研究がある。しかし、\RC についてはまだ調べられてこなかった。


% https://q.uiver.app/?q=WzAsNCxbMywwLCJcXHRleHR7XFxTTUMgfSJdLFszLDIsIlxcdGV4dHtcXFJDIH0iXSxbMCwyLCJcXHRleHR7XFxDQ30iXSxbMCwwLCJcXHRleHR7XFxBQyB9Il0sWzEsMCwiXFx0ZXh0e+WvvuinkuWwhOOCkuW/mOWNtH0iLDJdLFsyLDEsIlxcdGV4dHvntYLlr77osaHjgpLlv5jljbR9IiwyXSxbMiwzLCJcXHRleHR75a++6KeS5bCE44KS5b+Y5Y20fSJdLFszLDAsIlxcdGV4dHvntYLlr77osaHjgpLlv5jljbR9IiwyXSxbMSwyLCJcXExhbWJkYSIsMix7ImN1cnZlIjozfV0sWzAsMywiTCIsMix7ImN1cnZlIjozfV0sWzgsNSwiIiwxLHsibGV2ZWwiOjEsInN0eWxlIjp7Im5hbWUiOiJhZGp1bmN0aW9uIn19XSxbOSw3LCIiLDIseyJsZXZlbCI6MSwic3R5bGUiOnsibmFtZSI6ImFkanVuY3Rpb24ifX1dXQ==
\[\begin{tikzcd}
	{\text{\AC }} &&& {\text{\SMC }} \\
	\\
	{\text{\CC}} &&& {\text{\RC }}\\
	\arrow["{\text{対角射を忘却}}"', from=3-4, to=1-4]
	\arrow[""{name=0, anchor=center, inner sep=0}, "{\text{終対象を忘却}}"', from=3-1, to=3-4]
	\arrow["{\text{対角射を忘却}}", from=3-1, to=1-1]
	\arrow[""{name=1, anchor=center, inner sep=0}, "{\text{終対象を忘却}}"', from=1-1, to=1-4]
	\arrow[""{name=2, anchor=center, inner sep=0}, "\Lambda"', curve={height=18pt}, from=3-4, to=3-1]
	\arrow[""{name=3, anchor=center, inner sep=0}, "L"', curve={height=18pt}, from=1-4, to=1-1]
	\arrow["\dashv"{anchor=center, rotate=-90}, draw=none, from=2, to=0]
	\arrow["\dashv"{anchor=center, rotate=-92}, draw=none, from=3, to=1]
\end{tikzcd}\]

\topic{本論文では、\RC について調べる足掛かりとして、\CC との関係を調べることを試みた。}具体的には、\RC がしばしば\CC の\subRC として構成されることに着目し、逆に与えられた\RC がどのような\CC のよい\subRC となるのかを調べた。結果、次の事実を得た:任意の\RC に対し、それをよい\subRC に持つ\CC は存在すれば一意である。さらに全ての\RC に対して、そのような\CC が存在するかどうか、また存在すればそれがどのような\CC であるかを計算する方法を与えた。


\topic{本論文で主に扱うのは、与えられた\RC からそれをよい\subRC に持ちうる\CC を構成する普遍的な操作$\Lambda$である。}操作$\Lambda$は、あらゆる\RC $\R$に対して\CC $\Lambda(\R)$ 及び函手$\Eta_{\R}:\R \to \Lambda(\R)$の組$(\Lambda(\R), \Eta_{\R})$を1つずつ返すものである。この$\Eta_{\R}$は必ずしも単射的(即ち忠実かつ対象上本質的に単射)であるとは限らないが、もしも$\R$が何か\CC $\C$のよい\subRC として構成されているときには、必ず$\Lambda(\R)\simeq \C$である。特に、$\R$をよい\subRC に持つ\CC $\C$は同型を除いて一意的であることが従う。
\topic{さらにこの操作$\Lambda$は、\RC から\CC を構成する方法として普遍的である。}全ての\CC は自然な方法で\RC と見なせるが、この忘却操作に対して$\Lambda$は左随伴である。%また、この随伴はただの随伴ではなくreflectionである。\CC に対しては$\Lambda$を行っても不変であり、即ちこの忘却操作と$\Lambda$を連続して適用しても\CC たちの成す2-圏上の恒等函手に同型である。つまり、\CC の2-圏は\RC の2-圏のreflective subcategoryである。

\topic{$\Lambda\-$constructionの着想の説明をする。}
\RC から\CC を作る方法は「終対象を追加する」ものであるはずだから、\SMC から\AC を作る方法に一致するか少なくとも類似するはずである。後者は実際に知られていて、
\topic{対称モノイド圏から、\AC を作る(左)普遍的な構成が
\cite{HT12}の系として得られている(\cite{HT12}, \cite{huot2019universal})。}
この普遍的な構成、L-construction(named by \cite{huot2019universal})は、与えられた対称モノイド圏の射を余分に増やしたのち適切な同値関係で割る構成である。
特にモノイド単位対象への射を全て同一視するような同値関係で割るので、モノイド単位対象が終対象になる。標語的に言えば、L-constructionは「対称モノイド圏の単位対象を終対象にする構成」である。
%\topic{relevance categoryに施せば「対角射と終対象を備えた圏」即ちcartesian categoryになると期待される。}
\topic{L-constructionは\RC に施しても一般には\CC にはならない}。例えば$\Set^{\mono}$は素朴な方法で\RC の構造を持ち、$\Set^{\mono}$にL-constructionを施して得られる圏$L(\Set^{\mono})$は$\Set$と非同型であるが、$\Lambda(\Set^{\mono})\simeq\Set$である。$L\-$construction及び$\Lambda\-$constructionの普遍性を踏まえれば、$L(\Set^{\mono})$は\CC ではないことがわかる。
$L(\Set^{\mono})$については\cite{HT12}のThm4.4に類似物の言及があり、Tennent category(\cite{tennent1990semantical})との関連が示唆されている。

構成Lは\SMC のモノイド単位対象を終対象にするものであったが、上に述べたようにテンソル積を直積にするものではなかった。これに対して 構成$\Lambda$は、\RC を受け取り、そのモノイド単位対象を終対象に、テンソル積を直積にするものである。


本論文では、\RC (relevance category)の定義及び基本的な例を述べたのち、\RC の「\CC 化」操作である構成$\Lambda$を記述
%し、そのwell-defined性を証明
する。






\newpage










































\section{要旨に改変中の部分}
\topic{本論文では、\RC について調べる足掛かりとして、\CC との関係を調べることを試みた。}具体的には、\RC がしばしば\CC の\subRC として構成されることに着目し、逆に与えられた\RC がどのような\CC のよい\subRC となるのかを調べた。結果、次の事実を得た:任意の\RC に対し、それをよい\subRC に持つ\CC は存在すれば一意である。さらに全ての\RC に対して、そのような\CC が存在するかどうか、また存在すればそれがどのような\CC であるかを計算する方法を与えた。


\topic{本論文で主に扱うのは、与えられた\RC からそれをよい\subRC に持ちうる\CC を構成する普遍的な操作$\Lambda$である。}操作$\Lambda$は、あらゆる\RC $\R$に対して\CC $\Lambda(\R)$ 及び函手$\Eta_{\R}:\R \to \Lambda(\R)$の組$(\Lambda(\R), \Eta_{\R})$を1つずつ返すものである。この$\Eta_{\R}$は必ずしも単射的(即ち忠実かつ対象上本質的に単射)であるとは限らないが、もしも$\R$が何か\CC $\C$のよい\subRC として構成されているときには、必ず$\Lambda(\R)\simeq \C$である。特に、$\R$をよい\subRC に持つ\CC $\C$は同型を除いて一意的であることが従う。
\topic{さらにこの操作$\Lambda$は、\RC から\CC を構成する方法として普遍的である。}全ての\CC は自然な方法で\RC と見なせるが、この忘却操作に対して$\Lambda$は左随伴である。%また、この随伴はただの随伴ではなくreflectionである。\CC に対しては$\Lambda$を行っても不変であり、即ちこの忘却操作と$\Lambda$を連続して適用しても\CC たちの成す2-圏上の恒等函手に同型である。つまり、\CC の2-圏は\RC の2-圏のreflective subcategoryである。

\topic{$\Lambda\-$constructionの着想の説明をする。}
\RC から\CC を作る方法は「終対象を追加する」ものであるはずだから、\SMC から\AC を作る方法に一致するか少なくとも類似するはずである。後者は実際に知られていて、
\topic{対称モノイド圏から、\AC を作る(左)普遍的な構成が
\cite{HT12}の系として得られている(\cite{HT12}, \cite{huot2019universal})。}
この普遍的な構成、L-construction(named by \cite{huot2019universal})は、与えられた対称モノイド圏の射を余分に増やしたのち適切な同値関係で割る構成である。
特にモノイド単位対象への射を全て同一視するような同値関係で割るので、モノイド単位対象が終対象になる。標語的に言えば、L-constructionは「対称モノイド圏の単位対象を終対象にする構成」である。
%\topic{relevance categoryに施せば「対角射と終対象を備えた圏」即ちcartesian categoryになると期待される。}
\topic{L-constructionは\RC に施しても一般には\CC にはならない}。例えば$\Set^{\mono}$は素朴な方法で\RC の構造を持ち、$\Set^{\mono}$にL-constructionを施して得られる圏$L(\Set^{\mono})$は$\Set$と非同型であるが、$\Lambda(\Set^{\mono})\simeq\Set$である。$L\-$construction及び$\Lambda\-$constructionの普遍性を踏まえれば、$L(\Set^{\mono})$は\CC ではないことがわかる。
$L(\Set^{\mono})$については\cite{HT12}のThm4.4に類似物の言及があり、Tennent category(\cite{tennent1990semantical})との関連が示唆されている。

構成Lは\SMC のモノイド単位対象を終対象にするものであったが、上に述べたようにテンソル積を直積にするものではなかった。これに対して 構成$\Lambda$は、\RC を受け取り、そのモノイド単位対象を終対象に、テンソル積を直積にするものである。


本論文では、\RC (relevance category)の定義及び基本的な例を述べたのち、\RC の「\CC 化」操作である構成$\Lambda$を記述
%し、そのwell-defined性を証明
する。





対称モノイド圏論とは「対称モノイド圏とデカルト圏の中間構造たちを調べる分野」であると言うことができる。




\topic{対称モノイド圏論の視点で見れば、デカルト圏とは次のような構造だとみなせる:}デカルト圏とは、対称モノイド圏 $\C$と自然同型$\phi_{X,Y}:\C(-,X)\times\C(-,Y)\simeq\C(-,X\tensor Y)$の組$(\C,\phi)$のことである。
\topic{この同型を弱めて、対称モノイド圏$\C$によい自然変換$\C(-,X)\times\C(-,Y)\to\C(-,X\tensor Y)$を付加した構造や、対称モノイド圏$\C$によい自然変換$\C(-,X)\times\C(-,Y)\gets\C(-,X\tensor Y)$を付加した構造を考える。}これらはそれぞれ\topic{関連圏(relevance category)} / \topic{アフィン圏(affine category)}と呼ばれるものと一致し、対称モノイド圏とデカルト圏の中間構造として知られている(\cite{Pet02})。
関連圏 / アフィン圏 は、対称モノイド圏にそれぞれ「対角射」「終対象への標準射」を付加したものとして定義される。即ち、関連圏とは対称モノイド圏$\C$とよいモノイド自然変換$\Delta : Id_{\C}\to -\tensor -$の組であり、アフィン圏は対称モノイド圏$\C$であって単位対象が終対象であるようなものである(ただし、ここで函手$-\tensor-$は各対象$X$に$X\tensor X$を対応させる自己函手のことである)。
関連圏には例えば集合と単射の圏 $\Set^{\mono}$ などがある。アフィン圏には例えば凸空間の圏 $\mathrm{Conv}$ などがある。



\topic{\RC や\AC は、対称モノイド圏と\CC のよい中間構造と思える。}具体的には、対称モノイド圏に「対角射」を追加すれば\RC に、「終対象」を追加すれば\AC になるし、\CC は\RC に「終対象」を追加したものと見なせ、また\AC に「対角射」を追加したものとも見なせる。\topic{ゆえに、これら中間構造を調べることは対称モノイド圏と\CC の関係性を調べるのに有用である。}例えば、与えられた対称モノイド圏 $\M$から普遍的な\CC $CC[\M]$を構成する問題(直積を付加する問題)は、与えられた対称モノイド圏 $\M$から普遍的な\RC $RC[\M]$を構成する問題(対角射を付加する問題)と与えられた\RC $\R$から普遍的な\CC $CC[\R]$を構成する問題(終対象を付加する問題)とに分割できる。
実際\AC については、アフィン論理への関心も相まって多くの研究がある。しかし、\RC についてはまだ調べられてこなかった。


% https://q.uiver.app/?q=WzAsNCxbMywwLCJcXHRleHR7XFxTTUMgfSJdLFszLDIsIlxcdGV4dHtcXFJDIH0iXSxbMCwyLCJcXHRleHR7XFxDQ30iXSxbMCwwLCJcXHRleHR7XFxBQyB9Il0sWzEsMCwiXFx0ZXh0e+WvvuinkuWwhOOCkuW/mOWNtH0iLDJdLFsyLDEsIlxcdGV4dHvntYLlr77osaHjgpLlv5jljbR9IiwyXSxbMiwzLCJcXHRleHR75a++6KeS5bCE44KS5b+Y5Y20fSJdLFszLDAsIlxcdGV4dHvntYLlr77osaHjgpLlv5jljbR9IiwyXSxbMSwyLCJcXExhbWJkYSIsMix7ImN1cnZlIjozfV0sWzAsMywiTCIsMix7ImN1cnZlIjozfV0sWzgsNSwiIiwxLHsibGV2ZWwiOjEsInN0eWxlIjp7Im5hbWUiOiJhZGp1bmN0aW9uIn19XSxbOSw3LCIiLDIseyJsZXZlbCI6MSwic3R5bGUiOnsibmFtZSI6ImFkanVuY3Rpb24ifX1dXQ==
\[\begin{tikzcd}
	{\text{\AC }} &&& {\text{\SMC }} \\
	\\
	{\text{\CC}} &&& {\text{\RC }}\\
	\arrow["{\text{対角射を忘却}}"', from=3-4, to=1-4]
	\arrow[""{name=0, anchor=center, inner sep=0}, "{\text{終対象を忘却}}"', from=3-1, to=3-4]
	\arrow["{\text{対角射を忘却}}", from=3-1, to=1-1]
	\arrow[""{name=1, anchor=center, inner sep=0}, "{\text{終対象を忘却}}"', from=1-1, to=1-4]
	\arrow[""{name=2, anchor=center, inner sep=0}, "\Lambda"', curve={height=18pt}, from=3-4, to=3-1]
	\arrow[""{name=3, anchor=center, inner sep=0}, "L"', curve={height=18pt}, from=1-4, to=1-1]
	\arrow["\dashv"{anchor=center, rotate=-90}, draw=none, from=2, to=0]
	\arrow["\dashv"{anchor=center, rotate=-92}, draw=none, from=3, to=1]
\end{tikzcd}\]








\newpage










































\section{Reference}

%[Pet02] Petrić, Z. Coherence in Substructural Categories. Studia Logica 70, 271–296 (2002). https://doi.org/10.1023/A:1015186718090\\

%[HT12]"Monoidal indeterminates and categories of possible worlds"
%\url{https://www.sciencedirect.com/science/article/pii/S0304397512000163}\\



%[Lambek-Scott] Higher-Order Categorical Logic\\

%[Jacobs] Categorical Logic and Type Theory\\


\bibliography{Bibliography_Master_Thesis.bib} %hoge.bibの名前
\bibliographystyle{amsalpha} %参考文献出力スタイル(ここではamsalphaを使用)







\end{document}
